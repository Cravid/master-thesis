\chapter{Ziele und Grenzen}

\section{Zielsetzung}

Im Rahmen dieser Arbeit soll ein ROP Compiler entwickelt werden, welcher als Eingaben auf zur Verf�gung stehender Gadgets sowie vordefiniertem Code basierend eine Verkettung der Gadgets automatisiert durchf�hren soll. Die zur Verkettung ben�tigte Semantik sowie die zur Eingabe ben�tigten Formatregeln sollen ebenfalls innerhalb dieser Arbeit definiert und implementiert werden.\\

Ziel ist die Erstellung eines Proof-of-concept (PoC), welcher auf Basis einer zuvor ausgew�hlten spezifischen Prozessorarchitektur (X64, ARM32 oder ARM64) die grunds�tzliche Umsetzbarkeit demonstrieren und Ausgangspunkt f�r weitere Arbeiten bilden soll.\\

Au�erdem soll der Grad der Berechenbarkeit er�rtert und gezeigt werden, dass - abh�ngig vom zugrundeliegenden Zielprogramms, welches exploited werden soll - eine Touring-Vollst�ndigkeit nicht grunds�tzlich vorliegt.\\

Anhand verschiedener konstruierter sowie realer Beispiele soll der PoC evaluiert und die Umsetzbarkeit eines ROP Compilers nachgewiesen werden.\\

Zum Schluss werden die Ergebnisse mit anderen ROP Compilern verglichen und die Unterschiede in Performance sowie Qualit�t evaluiert.

\section{Abgrenzung}

In dieser Arbeit wird die Findung von ROP Gadgets nicht betrachtet, sondern durch bereits bestehende Projekte, welche innerhalb der Arbeit vorgestellt und referenziert werden sollen, zur weiteren Verarbeitung vorgegeben.\\

Des Weiteren wird sich in dieser Arbeit ausschlie�lich auf eine Prozessorarchitektur spezifiziert, da diese sich tiefgehend in den zugrundeliegenden Techniken unterscheiden. Ein universeller ROP Compiler kann daher in dieser Arbeit nicht entwickelt werden, bildet aber M�glichkeit f�r nachfolgende Projekte.