\chapter{Ziele und Grenzen}

\section{Zielsetzung}
\label{sec:Zielsetzung}

Im Rahmen dieser Arbeit soll ein ROP Compiler entwickelt werden, welcher als Eingaben auf zur Verf�gung stehender Gadgets sowie vordefinierten Zielcode (Payload) basierend eine Verkettung der Gadgets automatisiert durchf�hren soll. Der Compiler �bersetzt anschlie�end den Payload mittels der verf�gbaren Gadgets zu einer ROP Chain, welche semantisch gleich sind. Der zur Verkettung ben�tigte Algorithmus sowie die zur Eingabe ben�tigten Formatvorgaben sollen ebenfalls innerhalb dieser Arbeit definiert und implementiert werden.\\

Ziel ist die Erstellung eines Proof-of-concept (PoC), welcher auf Basis einer zuvor ausgew�hlten spezifischen Prozessorarchitektur (z.B. x64, ARM32 oder ARM64) die grunds�tzliche Umsetzbarkeit demonstrieren und Ausgangspunkt f�r weitere Arbeiten bilden soll.\\

Au�erdem soll der Grad der Berechenbarkeit er�rtert und evaluiert werden, ob mittels der verf�gbaren Gadgets eine ROP Chain f�r den gew�nschten Payload gefunden werden kann.\\

Anhand verschiedener konstruierter sowie realer Beispiele soll der PoC evaluiert und die Umsetzbarkeit eines ROP Compilers nachgewiesen werden.\\

Zum Schluss werden die Ergebnisse mit anderen ROP Compilern verglichen und die Unterschiede in Performance sowie Ergebnissen evaluiert.

\section{Abgrenzung}

In dieser Arbeit wird die Suche von ROP Gadgets selbst nicht betrachtet, sondern durch bereits bestehende Projekte, welche innerhalb der Arbeit vorgestellt und referenziert werden sollen, zur weiteren Verarbeitung vorgegeben.\\

Das genaue Format des Payload ist in dieser Arbeit noch zu definieren und k�nnte in Form von Pseudocode oder Konfigurationsvorgaben erfolgen. Eine universelle �bersetzung von Programmcode (z.B. C) kann innerhalb dieser Arbeit voraussichtlich nicht erfolgen, und w�rde Gegenstand weiterer Arbeiten bilden.\\

Aufgrund von grunds�tzlichen Unterschieden in den verschiedenen CPU-Architekturen, Befehlss�tzen sowie Executable File Formats (z.B. ELF, Mach-O, PE, usw.) wird sich in dieser Arbeit auf eine Architektur sowie ein File Format fokussiert. Ein universeller ROP Compiler f�r beliebige Architekturen kann innerhalb dieser Arbeit nicht entwickelt werden. Nachfolgende Arbeiten und Projekte k�nnen aber auf dieser Basis aufbauen.