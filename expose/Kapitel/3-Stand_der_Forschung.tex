\chapter{Stand der Forschung}

ROP und JOP sind Techniken, welche bereits seit Ende der 90er Jahre genutzt werden. Einer der wohl bekanntesten Exploits auf ROP Basis ist der return-into-libc Overflow Exploit \cite{ret2libc}.\\

F�r das Auffinden von Gadgets gibt es ebenfalls bekannte Projekte, welche in der Lage sind teilweise f�r verschiedene Prozessorarchitekturen Gadgets aus einer gegebenen Binary zu finden und anzeigen zu lassen. Ein Beispiel f�r solch ein Projekt w�ren ROPgadget \cite{JonathanSalwan/ROPgadget} oder Ropper \cite{sashs/Ropper}.\\

\emph{Q} \cite{schwartz2011q} war eine Forschungsarbeit der Carnegie Mellon University in Pittsburgh. In diesem Projekt wurden \enquote{semantic program verification techniques} genutzt, um einen gegebenen Exploit, welcher durch Sicherheitsmechanismen wie W$\oplus$X verhindert wird, in eine ROP Chain abzubilden, um den \enquote{Exploit zu h�rten} (exploit hardening). Ans�tze der Semantik zur Verkettung k�nnen eventuell f�r diese Arbeit interessant werden.\\

angrop \cite{angr/angrop} ist ein Tool zum Auffinden von ROP Gadgets. Au�erdem soll es ROP Chains automatisch verketten k�nnen. Es nutzt das Binary Analysis Framework angr \cite{angr}, welches von verschiedenen Forschern vom Computer Security Lab der UC Santa Barbara sowie dem SEFCOM der Arizona State University entwickelt wird. Gef�rdert wurde das Projekt von der U.S. Regierung. Die Ergebnisse wurden unter open source gestellt \cite{shoshitaishvili2016state}. angrop verfolgt den gleichen Ansatz von \emph{Q} mit einigen kleineren Abweichungen und Erweiterungen.\\

\emph{GENROP} \cite{branting2022rop} war Thema einer Masterthesis an der Link�ping University in Schweden. Das Thema von GENROP war sehr �hnlich zu dieser Arbeit und verfolgte den Ansatz Genetic Programming zum Verketten der ROP Gadgets zu nutzen. Das Ergebnis von GENROP war allerdings, dass die eigentlichen Kernkonzepte von Genetic Programming am Ende nicht mehr richtig genutzt werden konnten, um das Problem zu l�sen, da der Algorithmus zu sehr gelenkt werden musste. Genauere Einarbeitung und Analyse, welche Probleme bei GENROP auftraten, m�ssen noch erfolgen. Die Implementation des Projekts ist allerdings nicht �ffentlich verf�gbar.\\

Pure-Call Oriented Programming (PCOP) \cite{sadeghi2018pure} fokussiert sich auf Jump-oriented Programming (JOP) und somit Gadgets, welche in einer \code{call} Anweisung enden. Call Anweisungen werden vom Prozessor umgesetzt, indem die Speicheradresse der auf den Call direkt folgenden Anweisung auf den Memory Stack gelegt und eine \code{jmp} Anweisung zu der gew�nschten Speicheradresse ausgef�hrt wird. Dadurch, dass der Stack ver�ndert wird, ergeben sich Seiteneffekte, welche die Gadget Chain st�ren k�nnen. PCOP soll einige dieser Seiteneffekte beheben k�nnen. Au�erdem soll PCOP Touring-Vollst�ndigkeit besitzen, was bedeuten w�rde, dass beliebiger Code hierdurch emuliert werden k�nnte. Die genauen Ans�tze in diesem Artikel m�ssen im weiteren Verlauf noch herausgearbeitet werden.\\

Am Massachusetts Institue of Technology (MIT) wurde ein ROP Compiler \cite{stewart2015rop} erarbeitet, welcher die verf�gbaren Gadgets zuerst klassifiziert (basierend auf Logik aus \emph{Q} \cite{schwartz2011q}) und anschlie�end �ber die klassifizierten Gadgets iteriert. Au�erdem soll der vom MIT sogenannte \emph{Gadget Scheduler} auch bestimmte Speicherregister beobachten, sodass keine gegenseitigen �berschreibungen stattfinden sollen. Das Projekt soll auf konstruierten Beispielen sowie auf \code{rsync} als real-world Binary erfolgreich als PoC umgesetzt werden. Leider ist lediglich der Artikel und nicht die Codebasis �ffentlich verf�gbar.\\

Multi-Architecture JOP and ROP Chain Assembler (MAJORCA) \cite{nurmukhametov2021majorca} war ein Projekt des Moscow Institue of Physiscs and Technology, welches \enquote{spezielle Graphe zur Suche nach \code{mov} Ketten} zur Bildung der Gadget Chains genutzt hat. W�hrend sich dieses Projekt allerdings auf die x86 Architektur fokussiert, k�nnen Ideen und Ans�tze hieraus eventuell auch f�r diese Masterthesis interessant sein.\\

Es existieren somit einige Projekte, welche sich mit der automatisierten Generierung von ROP Chains befasst haben. Allerdings sind diese entweder nicht verf�gbar, funktionieren nicht in realistischen Umgebungen, arbeiten auf anderen Architekturen (meistens x86), oder sind nicht richtig ausgearbeitet.\\

Im Rahmen der weiteren Vor- und Recherchearbeiten sollen die einzelnen Techniken der vorgestellten sowie weiterer Arbeiten genauer herausgearbeitet, verglichen sowie f�r die Bildung der Semantik in dieser Arbeit herangezogen werden.