\chapter{Problemstellung und Motivation}

\emph{Return-oriented-programming (ROP)} und \emph{Jump-oriented-programming (JOP)} sind Angriffstechniken, welche genutzt werden, um Schadcode trotz vorhandener Sicherheitsmechanismen ausf�hren zu k�nnen. Dies erfordert im Zielprogramm eine Verwundbarkeit im Bereich Memory Corruption (z.B. Buffer Overflow, Use-After-Free (UAF), o.�.). Hierzu werden die im Zielprogramm vorhandenen Assembler-Instruktionen, welche sich bereits im Programmspeicher befinden, genutzt, um sogenannte \emph{ROP Gadgets} zu bilden.\\
Ein Gadget ist eine kleine Code Sequenz, welche eine bestimmte Aufgabe durchf�hrt, z.B. das Kopieren eines Speicherregisters in ein Anderes oder die Addition zweier Register. ROP oder JOP Gadgets enden typischerweise mit einer \code{return}, \code{jump} oder \code{call} Anweisung, welche die Ausf�hrung zu einer vorherig gespeicherten Stelle zur�ckbringen soll. Die Manipulation dieser Return-Addresse erlaubt es dem Angreifer Gadgets zu verketten, um eigene Operationen durchzuf�hren und ein gew�nschtes Ergebnis zu erzielen (z.B. das Ausf�hren von eigenem Code). Da lediglich Maschineninstruktionen genutzt werden, welche sich bereits im Speicher des Programms befinden, k�nnen viele Sicherheitsmechanismen, wie z.B. ASLR \footnote{ASLR (address space layout randomization) randomisiert die Speicherpositionen, sodass Adressbereiche nicht mehr vorhersehbar sind.}, NX / XD \footnote{NX-Bit (no-execute, AMD) bzw. XD (execute disable, Intel) aktivierte Prozessoren markieren Teile des Speichers als nicht-ausf�hrbar.}, W$\oplus$X \footnote{W$\oplus$X (write xor execute) sichert Prozess- oder Kernel-Speicherbereiche dadurch ab entweder schreib- oder ausf�hrbar zu sein, niemals beides gleichzeitig.} oder DEP \footnote{Data Execution Prevention (DEP) ist das Microsoft Windows Equivalent zu NX.}, umgangen werden.\\

Die Entwicklung eines ROP-basierten Exploit erfordert es bisher manuell ROP Gadgets zu verketten - d.h. Chains zu bilden - sowie den passenden Program Input zu finden, um �ber eine Memory Corruption die ROP Chain zu starten. Diese Prozesse durchzuf�hren ist nicht nur aus Angreifersicht von Interesse, sondern auch aus Sicht von Sicherheitsforschern wichtig, um m�gliche Exploits finden und anschlie�end beheben bzw. melden zu k�nnen.\\

Die manuelle Suche einer ROP Chain ist zeitintensiv und erfordert tiefgreifende Kenntnisse der zugrundeliegenden Architektur, des Zielprogramms sowie der Funktionsweise von Maschinenprozessen und Speicherverwaltung. An die durchf�hrende Person werden somit hohe Anforderungen gestellt, welche das eigentliche Auffinden von Schwachstellen erschwert.\\

Durch die Erstellung eines \emph{ROP Compilers} k�nnten diese Prozesse schrittweise automatisiert werden. Dies w�rde den zeitlichen Aufwand verringern sowie die M�glichkeit des Nachweises einer Schwachstelle mittels eines Exploits bzw. PoC an eine breitere Masse von Personen geben, um so Schwachstellen schneller finden, beheben als auch im Bereich der Strafverfolgung sowie Nachrichtendienste nutzen zu k�nnen.