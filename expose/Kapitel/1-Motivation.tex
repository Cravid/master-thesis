\chapter{Problemstellung und Motivation}

\emph{Return-oriented-programming (ROP)} und \emph{Jump-oriented-programming} Angriffstechniken, welche genutzt werde, um Schadcode trotz vorhandener Sicherheitsmechanismen ausf�hren zu k�nnen. Dies erfordert lediglich ein Zielprogramm, welches einen User Input zul�sst. Hierzu werden die im Zielprogramm vorhandenen Programmanweisungen, welche sich bereits im Program Memory befinden, genutzt, um sogenannte \emph{ROP Gadgets} zu bilden.\\
Ein Gadget ist ein kleiner Code Teil, welcher eine bestimmte Aufgabe durchf�hrt, z.B. das Kopieren eines Speicherregisters in ein Anderes oder die Addition zweier Register. ROP oder JOP Gadgets enden typischerweise mit einer \emph{return}, \emph{jump} oder \emph{call} Anweisung, welche die Ausf�hrung zu einer vorherig gespeicherten Stelle zur�ckbringen soll. Durch Manipulation dieser Return-Address erlaubt es dem Angreifer Gadgets zu verketten, um eigene Operationen durchzuf�hren und ein gew�nschtes Ergebnis zu erzielen. Da lediglich Maschineninstruktionen genutzt werden, welche sich bereits im Speicher des Programms bzw. einer Library befindet, k�nnen viele Sicherheitsmechanismen, wie z.B. NX oder W\textasciicircum X, umgangen werden.\\

Die Entwicklung von solchen Exploits erfordert es derzeit noch manuell solche ROP Gadget Chains sowie den passenden Program Input zu finden und zu verketten. Diese Prozesse durchzuf�hren ist nicht nur aus Angreifersicht von Interesse, sondern auch aus Sicht von Security Researchers (White Heads / Blue Teams) wichtig, um m�gliche Exploits finden und anschlie�end beheben bzw. melden zu k�nnen.\\

Das manuelle Finden dieser ROP Chains ist hoch zeitintensiv und erfordert tiefgreifende Kenntnisse der zugrundeliegenden Hardware, Architektur, des Zielprogramms sowie der Funktionsweise von Maschinenprozessen und Speicherverwaltung. An den Security Researcher werden somit hohe Anforderungen gestellt, welche das eigentliche Auffinden von Schwachstellen erschwert.\\
Durch die Erstellung eines \emph{ROP Compilers} k�nnten diese Prozess schrittweise automatisiert werden. Dies w�rde den zeitlichen Aufwand verringern sowie die M�glichkeit der Suche von Schwachstellen an eine breitere Masse von Personen geben, um so gefundene Schwachstellen schneller finden und beheben zu k�nnen.